\section{boxes and columns}
 \frame{\sectionpage}
 
\begin{frame}{Box}
    \begin{center}
        \textcolor{yellow}{\fbox{phrase inside box}} 

    \bigskip    
    
    \fbox{\parbox{\textwidth}{A big box\\ \[ 
    \{R^n_{\alpha}(0) \; | \; n \in \mathbb{N}\} = \{n\alpha \; \mathrm{mod}\;1 \; | \; n \in \mathbb{N}\} \]
    é denso em \([0,1)\). }}
 \end{center}
\end{frame}
    
\begin{frame}{Two Columns entire page}
\small

\begin{columns}
\begin{column}{0.5\textwidth}
\tiny \pause \textcolor{yellow}{Obs:} \(\alpha \eqdef \log b \in \mathbb{R}\backslash\mathbb{Q}\)
\small \pause 
   \begin{align*}
    R_\alpha \colon [0,1) &\longrightarrow [0,1) \\
                      x   &\longmapsto x + \alpha \; \mathrm{mod}\;1           
\end{align*}
\pause 
\small 
Here we can write some text Here we can write some text Here we can write some text Here we can write some text Here we can write some text  Here we can write some text  Here we can write some text
\end{column}
\begin{column}{0.5\textwidth}  %%<--- here
\pause 
\[R^n_\alpha(x) \eqdef R_\alpha \overbrace{\circ \ldots \circ }^{n} R_\alpha(x)\]

\pause 

Here we can write some text Here we can write some text Here we can write some text Here we can write some text Here we can write some text  Here we can write some text  Here we can write some text

\bigskip 

\footnotesize 
\textcolor{yellow}{\fbox{\parbox{\textwidth}{Question??????????? tell me if you want}}} 
\bigskip

the answer is 
\pause 
\textcolor{green2}{YES!!!!} \textcolor{green2}{because that that and that} or..

\bigskip 

The answer is \textcolor{red2}{NO!!!!} \textcolor{red2}{because that that and that}

\end{column}
\end{columns}

\end{frame}
    
    
\begin{frame}{Table and minipage}

\begin{center}
\begin{tabular}{|l|l|l|l|l|l|l|l|l|l|l|l|l|}
\hline
\(n\) & 1  & 2 & 3 & 4 & 5 & 6 & 7 & 8 & 9 & 10 & 11 & \ldots       \\ \hline
\(2^n\) & \textcolor{yellow}{2} & \textcolor{yellow}{4} & \textcolor{yellow}{8} & \textcolor{yellow}{1}6 & \textcolor{yellow}{3}2 & \textcolor{yellow}{6}4 & \textcolor{yellow}{1}28 &  \textcolor{yellow}{2}56 & \textcolor{yellow}{5}12 & \textcolor{yellow}{1}024 & \textcolor{yellow}{2}048 & \ldots    \\ \hline
\end{tabular}
\end{center}

\pause 

\bigskip

\begin{center}
\textcolor{yellow}{\fbox{o dígito 1 é mais frequente que o dígito 3?}}
\bigskip

\pause \tiny Spoiler: \textcolor{green2}{YES}.

\end{center}


\pause 
\bigskip
\small 
\
\pause 
\begin{minipage}{0.47\textwidth}
    Um conjunto de números satisfaz a \emph{\textcolor{yellow}{lei de Benford}} se o primeiro dígito \(d \in \{1,2,3,4,5,6,7,8,9\} \) ocorre com a seguinte proporção 
\end{minipage}
\begin{minipage}{0.47\textwidth}
    \centering  \textcolor{yellow}{\(P(d) = \log\bigg(1+ \frac{1}{d}\bigg) \)} 
\end{minipage}
\end{frame}

