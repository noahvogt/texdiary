%%%%%%%%%%%%%%%%%%%%%%%%%%%%%%%%%%%%%%%%%%%%%%%%%%%%%%%%%%%%%%%%%%%%%%%%%%%%%%%%%%%
%% This project aims to create the UFC template for presentation.                %%
%% author: Maurício Moreira Neto - Doctoral student in Computer Science (MDCC)   %%
%% contacts:                                                                     %%
%%    e-mail: maumneto@ufc.br                                                    %%
%%    linktree: https://linktr.ee/maumneto                                       %%
%%%%%%%%%%%%%%%%%%%%%%%%%%%%%%%%%%%%%%%%%%%%%%%%%%%%%%%%%%%%%%%%%%%%%%%%%%%%%%%%%%%
\documentclass{libs/ufc_format}
% Inserting the preamble file with the packages
\usepackage{amsthm}
\usepackage{amsfonts}
\usepackage{amsmath}
\usepackage{amssymb}
\usepackage{mathtools}
\usepackage{bbm}
\usepackage{pgfplots}
\usepackage{tikz}
\usepackage{physics}
\usepackage{calligra}
\usepackage{csquotes}
\usepackage{tensor}
\usepackage[thicklines]{cancel}
\usepackage{tcolorbox}
\usepackage{pstricks}
\usepackage[backend=biber, bibstyle=nature, sorting=nty, citestyle=numeric-comp]{biblatex} %Custom bibliography
    \addbibresource{bib.bib} %Load references


\DeclareMathAlphabet{\mathcalligra}{T1}{calligra}{m}{n}
\DeclareFontShape{T1}{calligra}{m}{n}{<->s*[2.2]callig15}{}
\newcommand{\scriptr}{\mathcalligra{r}\,}
\newcommand{\boldscriptr}{\pmb{\mathcalligra{r}}\,}
\def\rc{\scriptr}
\def\brc{\boldscriptr}
\def\hrc{\hat\brc}
\newcommand{\ie}{\emph{i.e.}} %id est
\newcommand{\eg}{\emph{e.g.}} %exempli gratia
\newcommand{\rtd}[1]{\ensuremath{\left\lfloor #1 \right\rfloor}}
\newcommand{\dirac}[1]{\ensuremath{\delta \left( #1 \right)}}
\newcommand{\diract}[1]{\ensuremath{\delta^3 \left( #1 \right)}}
\newcommand{\e}{\ensuremath{\epsilon_0}}
\newcommand{\m}{\ensuremath{\mu_0}}
\newcommand{\V}{\ensuremath{\mathcal{V}}}
\newcommand{\prnt}[1]{\ensuremath{\left(#1\right)}} %parentheses
\newcommand{\colch}[1]{\ensuremath{\left[#1\right]}} %square brackets
\newcommand{\chave}[1]{\ensuremath{\left\{#1\right\}}}  %curly brackets
\newcommand\eqdef{\stackrel{\mathclap{\normalfont \tiny\mbox{\textrm{def}}}}{=}}
\useoutertheme{infolines}
\useinnertheme{rectangles}
\usefonttheme{professionalfonts}


\definecolor{blue2}{HTML}{045FB4}
\definecolor{green2}{HTML}{46C235}
\definecolor{red2}{HTML}{EE4848}
\definecolor{violet2}{HTML}{A647E5}
\definecolor{orange2}{HTML}{FF7425}
\definecolor{darkred}{HTML}{5C2020}
\definecolor{gray}{HTML}{303030}
\definecolor{yellow}{HTML}{f0be52}
\definecolor{lightdarkgold}{HTML}{EEBC1D}

\renewcommand{\CancelColor}{\color{darkred}}

\makeatletter
\newcommand{\mybox}[1]{%
  \setbox0=\hbox{#1}%
  \setlength{\@tempdima}{\dimexpr\wd0+13pt}%
  \begin{tcolorbox}[colback=darkred,colframe=darkred,boxrule=0.5pt,arc=4pt,
      left=6pt,right=6pt,top=6pt,bottom=6pt,boxsep=0pt,width=\@tempdima]
    \textcolor{yellow}{#1}
  \end{tcolorbox}
}
\makeatother


\pgfplotsset{my style/.append style={axis x line=middle, axis y line=
middle, xlabel={$x$}, ylabel={$y$}, axis equal }}


\usecolortheme[named=darkred]{structure}
\usecolortheme{sidebartab}
\usecolortheme{orchid}
\usecolortheme{whale}
\setbeamercolor{titlelike}{parent=structure, bg=structure, fg=white}
\setbeamercolor{section in toc}{fg= white}
\setbeamercolor{subsection in toc}{fg= white}
%\setbeamercolor*{sidebar}{fg=red2,bg=gray!15!white}

\setbeamercolor{item projected}{bg=yellow, fg = darkred}
\setbeamertemplate{enumerate items}[default]
\setbeamertemplate{navigation symbols}{}
\setbeamercolor{local structure}{fg=yellow}

\setbeamercolor{alerted text}{fg=white}
\setbeamercolor{block title}{bg = blue2}
\setbeamercolor{block title alerted}{bg=red2}
\setbeamercolor{block title example}{bg=green2}
\setbeamercolor{background canvas}{bg=gray}
\setbeamercolor{normal text}{bg=gray,fg=white}


\setbeamertemplate{footline}
        {
      \leavevmode%
      \hbox{%
      \begin{beamercolorbox}[wd=.333333\paperwidth,ht=2.25ex,dp=1ex,center]{author in head/foot}%
        \usebeamerfont{author in head/foot}\insertshortauthor~~(\insertshortinstitute)
      \end{beamercolorbox}%
      \begin{beamercolorbox}[wd=.333333\paperwidth,ht=2.25ex,dp=1ex,center]{title in head/foot}%
        \usebeamerfont{title in head/foot}\insertshorttitle
      \end{beamercolorbox}%
      \begin{beamercolorbox}[wd=.333333\paperwidth,ht=2.25ex,dp=1ex,center]{date in head/foot}%
        \usebeamerfont{date in head/foot}\insertshortdate{}%\hspace*{2em}

    %#turning the next line into a comment, erases the frame numbers
        %\insertframenumber{} / \inserttotalframenumber\hspace*{2ex} 

      \end{beamercolorbox}}%
      \vskip0pt%
    }


\setbeamertemplate{blocks}[rectangle]
\setbeamercovered{dynamic}




%\setbeamercolor{author}{fg=yellow}
%\setbeamercolor{title}{fg = yellow}
%\setbeamerfont{title}{size=\Large, series=\bfseries}
%\setbeamerfont{author}{size=\footnotesize}
%\setbeamerfont{date}{size=\small}


\setbeamertemplate{section page}
{
	\begin{centering}
		\begin{beamercolorbox}[sep=27pt,center]{part title}
			\usebeamerfont{section title}\insertsection\par
			\usebeamerfont{subsection title}\insertsubsection\par
		\end{beamercolorbox}
	\end{centering}
}





%\setbeamertemplate{subsection page}
%{
%	\begin{centering}
%		\begin{beamercolorbox}[sep=12pt,center]{part title}
%			\usebeamerfont{subsection title}\insertsubsection\par
%		\end{beamercolorbox}
%	\end{centering}
%}

\newcommand{\hlight}[1]{\colorbox{violet!50}{#1}}
\newcommand{\hlighta}[1]{\colorbox{darkred!50}{#1}}

% Inserting the references file
\bibliography{references.bib}

% Title
\title[\textit{snailmail}]{\textbf{Eine Email-Client-App entwickeln}}
% Subtitle
\subtitle{snailmail}
% Author of the presentation
\author{Noah Vogt und Simon Hammer}

% Institute's Name
\institute[]{
    % university name
    \ufc
}
% date of the presentation
\date{5 Februar 2022}


%%%%%%%%%%%%%%%%%%%%%%%%%%%%%%%%%%%%%%%%%%%%%%%%%%%%%%%%%%%%%%%%%%%%%%%%%%%%%%%%%%
%% Start Document of the Presentation                                           %%               
%%%%%%%%%%%%%%%%%%%%%%%%%%%%%%%%%%%%%%%%%%%%%%%%%%%%%%%%%%%%%%%%%%%%%%%%%%%%%%%%%%
\begin{document}
% insert the code style
%%%%%%%%%%%%%%%%%%%%%%%%%%%%%%%%%%%%%%%%%%%%%%%%%%%%%%%%%%%%%%%%%%%%%%%%%%%%%%%%%%%
%% This file contains the style of the codes show in slides.                     %%
%% The package used is listings, but it possible to used others.                 %%
%%%%%%%%%%%%%%%%%%%%%%%%%%%%%%%%%%%%%%%%%%%%%%%%%%%%%%%%%%%%%%%%%%%%%%%%%%%%%%%%%%%

% color used in the code style
\definecolor{codegreen}{rgb}{0,0.6,0}
\definecolor{codegray}{rgb}{0.5,0.5,0.5}
\definecolor{codepurple}{rgb}{0.58,0,0.82}
\definecolor{codebackground}{rgb}{0.95,0.95,0.92}

% style of the code!
\lstdefinestyle{codestyle}{
    backgroundcolor=\color{codebackground},   
    commentstyle=\color{codegreen},
    keywordstyle=\color{magenta},
    numberstyle=\tiny\color{codegray},
    stringstyle=\color{codepurple},
    basicstyle=\ttfamily\footnotesize,
    frame=single,
    breakatwhitespace=false,         
    breaklines=true,                 
    captionpos=b,                    
    keepspaces=true,                 
    numbers=left,                    
    numbersep=5pt,                  
    showspaces=false,                
    showstringspaces=false,
    showtabs=false,                  
    tabsize=2,
    title=\lstname 
}

\lstset{style=codestyle}


%% ---------------------------------------------------------------------------
% First frame (with tile, subtitle, ...)
\begin{frame}{}
    \maketitle
\end{frame}

%% ---------------------------------------------------------------------------
% Second frame
\begin{frame}{Sumário}
    \begin{multicols}{2}
        \tableofcontents
    \end{multicols}
\end{frame}

%% ---------------------------------------------------------------------------
% This presentation is separated by sections and subsections
\section{Seção I}
\begin{frame}{Explicações}
    % itemize
    Este é um template que pode ser utilizado para:
    \begin{itemize}
        \item Apresentação de Trabalhos Acadêmicos
        \item Apresentação de Disciplinas
        \item Apresentações de Teses e Dissertações
    \end{itemize}

    \vspace{0.4cm} % vertical space
    
    % enumeration
    Para utilizar este template corretamente é importante que:
    \begin{enumerate}
        \item Tenha conhecimento mínimo sobre LaTeX
        \item Ler os comentários no template (explicações)
        \item Ler o README.md (documentação)
    \end{enumerate}

    \vspace{0.2cm}

    \example{Este é um texto de exemplo!} \emph{Texto de Ênfase!}
\end{frame}

%% ---------------------------------------------------------------------------
\subsection{Subseção I}
\begin{frame}{Criando Blocos}
    % Blocks styles
    \begin{block}{Bloco Padrão}
        Texto do corpo do bloco.
    \end{block}

    \begin{alertblock}{Bloco de Alerta}
        Texto do corpo do bloco.
    \end{alertblock}

    \begin{exampleblock}{Bloco de Exemplo}
        Texto do corpo do bloco.
    \end{exampleblock}   
\end{frame}

%% ---------------------------------------------------------------------------
\subsection{Subseção II}
\begin{frame}{Criando Caixas}
    \successbox{testando o success box}

    \pause

    \alertbox{testando o alert box}

    \pause

    \simplebox{testando o simple box}
\end{frame}

%% ---------------------------------------------------------------------------
\subsection{Subseção III}
\begin{frame}{Criando Algoritmos (Pseudocódigo)}
    \begin{algorithm}[H]
        \SetAlgoLined
        \LinesNumbered
        \SetKwInOut{Input}{input}
        \SetKwInOut{Output}{output}
        \Input{x: float, y: float}
        \Output{r: float}
        \While{True}{
          r = x + y\;
          \eIf{r >= 30}{
           ``O valor de $r$ é maior ou iqual a 10.''\;
           break\;
           }{
           ``O valor de $r$ = '', r\;
          }
         } 
         \caption{Algorithm Example}
    \end{algorithm}
\end{frame}

%% ---------------------------------------------------------------------------

\begin{frame}{Inserindo Algoritmos}
    \lstset{language=Python}
    \lstinputlisting[language=Python]{code/main.py}
\end{frame}

%% ---------------------------------------------------------------------------
\begin{frame}{Inserindo Algoritmos}
    \lstinputlisting[language=C]{code/source.c}
\end{frame}

%% ---------------------------------------------------------------------------
\begin{frame}{Inserindo Algoritmos}
    \lstinputlisting[language=Java]{code/helloworld.java}
\end{frame}

%% ---------------------------------------------------------------------------
\begin{frame}{Inserindo Algoritmos}
    \lstinputlisting[language=HTML]{code/index.html}
\end{frame}

%% ---------------------------------------------------------------------------
% This frame show an example to insert multicolumns
\section{Multicolunas}
\begin{frame}{Seção II - Multicolunas}
    \begin{columns}{}
        \begin{column}{0.5\textwidth}
            \justify
            É possível colocar mais de uma coluna utilizando os comandos de $\backslash$begin\{column\}\{\} e $\backslash$end\{column\}
        \end{column}
        \begin{column}{0.5\textwidth}
            \justify
            Porém, o espaçamento deve ser proporcional entre as colunas para que estas colunas não entrem em coflito. O espaçamento é dado pelo segundo argumento do $\backslash$begin.
        \end{column}
    \end{columns}    
\end{frame}

%% ---------------------------------------------------------------------------
%This frame show an example to insert figures
\section{Imagens}
\begin{frame}{Seção III - Figures}
    \begin{figure}
        \centering
        \caption{Emblema da UFC.}
        \includegraphics[scale=0.3]{libs/emblemufc.pdf}
        \source{Obtido pelo site oficial da UFC \cite{siteufc} \cite{einstein}}
        \label{fig:ufc_emblem}
    \end{figure}
\end{frame}

%% ---------------------------------------------------------------------------
% Reference frames
\begin{frame}[allowframebreaks]
    \frametitle{Referências}
    \printbibliography
\end{frame}

%% ---------------------------------------------------------------------------
% Final frame
\begin{frame}{}
    \centering
    \huge{\textbf{\example{Obrigado(a) pela Atenção!}}}
    
    \vspace{1cm}
    
    \Large{\textbf{Contato:}}
    \newline
    \vspace*{0.5cm}
    \large{\email{usuario@dominio}}
\end{frame}

\end{document}
