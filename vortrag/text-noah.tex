%standard
\documentclass[a4paper,10pt]{scrartcl}
\usepackage[utf8]{inputenc}
\usepackage[ngerman]{babel}
%\selectlanguage{english}

%font
\usepackage{tgheros}
\usepackage{mathptmx}
% use sans serif font
% \renewcommand{\familydefault}{\sfdefault}

\usepackage[hyphens]{url}
\usepackage{hyperref}

\usepackage{multicol}

%pictures
\usepackage{graphicx}
\usepackage{wrapfig}

%tables
\renewcommand{\arraystretch}{1.4}
\usepackage{multirow}

%special characters
\usepackage[official]{eurosym}
\newcommand\tab[1][1cm]{\hspace*{#1}}

%geometry
\usepackage{geometry}
\geometry{ headsep=20pt,
 headheight=20pt,
 left=21mm,
 top=15mm,
 right=21mm,
 bottom=15mm,
 footskip=20pt,
 includeheadfoot}

%header
\usepackage{datetime}
\newdateformat{dmy}{%
  \THEDAY.~\monthname[\THEMONTH] \THEYEAR}
\usepackage{fancyhdr}
\pagestyle{fancy}
\lhead{Noah J. Vogt}
\chead{}
\rhead{\dmy\today}
\lfoot{}
\cfoot{Gymnasium Kirschgarten}
\rfoot{}

%indent
\renewcommand{\footrulewidth}{.4pt}
\setlength{\parindent}{0em}

\begin{document}

\begin{multicols}{3}

\section{Maturvortrag}

\subsubsection*{Begrüssung + TOC (Noah)}

\subsubsection*{Motivation (Simon)}

\subsubsection*{Ziele (Noah)}

Die App soll die Basisfunktionen eines klassischen Email Clients
erfüllen. Dazu gehören das Lesen, Schreiben, Empfangen und Versenden von
Emails, das Öffnen und Anfügen von Anlagen, die Setzung einer
Email-Signatur und das Erstellen und Speichern von Entwürfen.\\

Ebenso soll es einen Account Manger geben, welcher dem Nutzer
ermöglicht, sich in der App mit mehreren Emailkonten anzumelden und
zwischen diesen zu wechseln.\\

Die App soll auch \emph{«Mobil und Modern»} wirken, also gut auf eine
Mobilplattform angepasst werden. Deshalb soll sie über Pushnachrichten
und ein visuell ansprechendes User Interface verfügen.\\

Auch soll ein Einstellungsfenster erstellt werden, wo der Nutzer
verschiedene Verhalten der App anpassen kann.\\

Grundsätzlich soll unsere App \emph{Free Software} sein und so
programmiert werden, dass sie alle nötigen Grundfunktionen für einen
Email Client auf dem Smartphone beinhaltet, aber schneller starten soll
als die Apps der Konkurrenz, weniger Speicherplatz und Ressourcen
verbraucht und nicht mit unnötigen Funktionen überladen sein soll. Das
Motto ist hier: \emph{schnell, frei und simpel}.

\subsubsection*{Inspiration Design (Noah)}

Wir haben uns die verschiedenen Email Clients auf Android angeschaut und
viel gutes und schlechtes gesehen. Es scheint als seien diese aus 70\%
guten und 30\% schlechten Sachen zusammengesetzt. Wenn man diese
verschiedenen Designprinzipien kombiniert, könnte man eine aus unseren
Augen ziemlich passable Email App erschaffen. \\

So gefiel uns das responsive und für die Mobilnutzung ziemlich
ausgeklügelte Touch Interface bei Gmail, doch fanden wir es unnötig
überladen und langsam.\\

Während k9-mail ein schnelles UI (User Interface)
bietet, so empfanden wir die grafische Oberfläche als sehr undurchdacht
konzipiert. Bei Fairmail war das Interface dermassen überladen, dass wir
gar nicht weiter nach anderen, möglichen Vorteilen gesucht haben.

\subsubsection*{Demo (Video)}

\subsubsection*{Was ist alles drin (Simon)}

\subsubsection*{App-Struktur (Simon)}

\paragraph{Database (Noah)}

Eine Datenbank (engl. database) ist allgemein eine organisierte
Ansammlung von strukturierter Information oder Daten, typischerweise
digital auf einem Computersystem gespeichert, verwaltet und ausgewertet.\\

Datenbanken sind oft sehr ähnlich aufgebaut wie die für Endnutzer
entwickelten graphischen Tabellenkalkulationsprogramme wie z.B. Google
Docs oder Libre Office.

\paragraph{Database (Simon)}

\paragraph{Server Connection (Noah)}

Vereinfacht funktioniert der Versand von Emails in diesem Diagramm: Ein
Nutzer, der eine Email versenden will, interagiert mit seinem
Mail-Client und gibt durch ihn den Befehl, die Email zu versenden.\\

Der
Email Client verschickt die Mail an den SMTP Server des sendenden
Nutzers, wo dieser zum SMTP Server des empfangenden Nutzers gelangt, von
dort aus zu seinem IMAP oder POP3 Server. Wenn der Mail Client des
Empfängers eine Anfrage an seinen SMTP / POP 3 macht, kann er diese
einlesen und der Nutzer kann nun seine neue Email lesen.\\\\\\

\paragraph{Material Design (Noah)}

Material ist ein Framework, welche dem App Entwickler ermöglicht,
Fenster, Knöpfe, Texte und Texteingabefelder einzubinden, und das
visuell ansprechender als die standardmässigen Android Bibliotheken.\\

Dieses GUI-Framework ist sehr beliebt und wird oft auch in Google Apps
verwendet, was dem neuen Nutzer von \emph{snailmail} ein familiäre
Benutzererfahrung bescheren soll.

\subsubsection*{Bug (Simon)}

\subsubsection*{Resultate: Vergleich mit ursprünglichen Zielen (Noah)}

Das User Interface ist erfreulich gut im Einklang mit den ursprünglichen
Zielen: Es ist simpel und ohne unnötigen Schnickschnack. Die
Nutzerfreundlichkeit ist ebenfalls zufriedenstellend.\\

Unsere App sollte ja \emph{Free Software} werden, inklusive
Bibliotheken. Doch da wir uns noch nicht so gut auskannten mit Gradle,
schlich sich eine nicht-freie Library namens \emph{chaquopy} ein. Diese
werden wir noch entfernen.\\

Die wichtigsten Funktionen der App wurden erreicht, es können Emails
geschrieben und gelesen werden, es bestehen verschiedene Mailboxen, usw.
Gewisse Features, wie Pushnachrichten oder eine Suchfunktionen, fehlen
noch ganz. Funktionen wie die Einstellungen oder das Synchronisieren mit
dem Mailserver sind noch nicht ganz fertiggestellt. \\

Insgesamt haben wir also doch das meiste und wichtige erreichen können,
was ein Email Client auf einem Smartphone so haben muss. Das motiviert
uns natürlich für die weitere Entwicklung und Vollendung des Projekts
\emph{snailmail}.\\\\\\

\subsubsection*{Was haben wir gelernt (Simon)}

\subsubsection*{Persönliche Meinung (Noah)}

Die meisten Probleme die wir während der Programmierarbeit hatten, waren
primär zurückzuführen auf unsere fehlende Erfahrung mit der
Android-Entwicklungsumgebung.\\

Auch kannten wir noch wenige Java Libraries, weshalb uns vor allem am
Anfang die Suche nach passenden Bibliotheken einige Zeit kostete.\\\\\\\\

Insgesamt kann ich von mir aus sagen, dass mich das ganze Projekt
weitergebracht hat, die ersten Erfahrungen mit der App Programmierung
waren es wert und ich sehe sie persönlich und beruflich rein zum
Vorteil.\\

\subsubsection*{Persönliche Meinung (Simon)}

\subsubsection*{Abschluss (Simon)}

\end{multicols}

\end{document}
