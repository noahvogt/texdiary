\begin{center}
\section*{\month}
\end{center}


\def\day{\textit{Juli 14st, 2021 Simon}}
\def\weekday{\textit{Wednesday}}
\subsection*{\weekday, \day}

I'm reading a book right now about Database and I'm making the setting Fragment. The Recyclerviewer is a little bit annoying 
because I don't want to make a pop-up Window but I don't know how to get rid of the Recyclerviewer. And I'm not sure in what design 
I want to make the Settings. Also i should borrow a Book for Interface Standarts.
\\
https://source.android.com/devices/tech/settings/settings-guidelines
\\
This is a guidline to how to make Settings but I'm not sure if this should be our Desingtype


\def\day{\textit{Juli 17st, 2021 Simon}}
\def\weekday{\textit{Saturday}}
\subsection*{\weekday, \day}

The last days I added a Settings Fragment. You can see through because there is no Background and when you press the back button on the phone
the App closes. So you can't close the Settings. Also I think we shoul make the Settings at the end because we don't have 
all functions. 

With android.R.id.content in this use

getSupportFragmentManager()
                        .beginTransaction()
                        .replace(android.R.id.content, new SettingsFragment())
                        .commit();

I can replace an old Fragment, without knowing the ID of it, with the new one in de SettingsFragment() class.


\def\day{\textit{Juli 19st, 2021 Simon}}
\def\weekday{\textit{Monday}}
\subsection*{\weekday, \day}

Today I had success in adding settings to the app. 
I used this Tutorial an added the .SettingsAvtivity to the AndroidManifesf with this line 

<activity android:name=".ui.home.SettingsActivity"/>

https://www.geeksforgeeks.org/how-to-implement-preferences-settings-screen-in-android/

And I put every raw text into the String file. 
I wanted to do the same with the dimens but we are not shure how we make the default layout sizes 
so i thought i might leave it for now.


