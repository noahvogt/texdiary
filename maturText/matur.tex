% standard
\documentclass[a4paper,12pt]{article}
\usepackage[utf8]{inputenc}
\usepackage[ngerman]{babel}

% geometry
\usepackage{geometry}
\geometry{ headsep=20pt,
headheight=20pt,
left=21mm,
top=15mm,
right=21mm,
bottom=15mm,
footskip=20pt,
includeheadfoot}

% header and footer
\usepackage{datetime}
\newdateformat{dmy}{%
\THEDAY.~\monthname[\THEMONTH] \THEYEAR}
\usepackage{fancyhdr}
\pagestyle{fancy}
\lhead{Noah Vogt \& Simon Hammer}
\chead{}
%\rhead{\dmy\today}
\lfoot{}
\cfoot{Gymnasium Kirschgarten}
\rfoot{Seite \thepage}
\renewcommand{\footrulewidth}{.4pt}

% fix figure positioning
\usepackage{float}

% larger inner table margin
\renewcommand{\arraystretch}{1.4}

% no paragraph indent
\setlength{\parindent}{0em}

% graphics package
\usepackage{graphicx}

\usepackage{multicol}

% use sans serif font
\usepackage{tgheros}
\usepackage{mathptmx}

% don't even ask what this is for, I have no idea (noah)
\usepackage{bm} %italic \bm{\mathit{•}}
\usepackage[hang]{footmisc}
\usepackage{siunitx}
\usepackage[font={small,it}]{caption}
\sisetup{locale = DE, per-mode = fraction, separate-uncertainty,   exponent-to-prefix, prefixes-as-symbols = false, scientific-notation=false
}
\newcommand{\ns}[4]{(\num[scientific-notation=false]{#1}\pm\num[scientific-notation=false]{#2})\cdot\num[]{e#3}\si{#4}}

% show isbn in bibliography
\usepackage{natbib}

\begin{document}

\begin{titlepage}

\vspace*{1cm}
	\centering
	
	{\huge\bfseries Eine Email-Client-App entwickeln \par}
	\vspace{0.5cm}
	{\Large Noah Vogt \& Simon Hammer\par}
	\vspace{17cm}

	{\large Geschrieben im Jahr 2021\par}
	
\end{titlepage}

\tableofcontents
\pagebreak

\section*{Ideenfindung}







\section*{Werkzeuge}

Aufgrund dessen, dass ein umfassendes Programm entstehen soll, wird auch gebrauch von einigen anderen Programmen gemacht. In den nachfolgenden Seiten wird beschrieben welche Programme genutz
werden, wieso diese ausgewählt wurden und wie der Umgang mit Ihnen war.  





\subsection*{Git/GitHub}


Git und GitHub sind wohl die wichtigsten Programme die genutz wurden. Sie sind Systeme, welche Fileordner (repository) verwalten können und sie für mehrere Computer zur verfügung stellen, 
wobei sie sehr viele praktische funktionen mit sich bringen. Mit Git können repositorys local auf Computer oder Hardware geteilt werden, mit GitHub könne die repositorys auch 
über das Internet geteilt werden. Der einfachheitshalber wird nicht zwischen Git und GitHub unterschieden. 







\subsection*{AndroidStudio}
\subsection*{Librarys}

\end{document}
