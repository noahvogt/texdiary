% standard
\documentclass[a4paper,11pt]{article}
\usepackage[utf8]{inputenc}
\usepackage[ngerman]{babel}

\usepackage{titlesec}

\setcounter{secnumdepth}{5}
\setcounter{tocdepth}{5}

\titleformat{\paragraph}
{\normalfont\normalsize\bfseries}{\theparagraph}{1em}{}
\titlespacing*{\paragraph}
{0pt}{3.25ex plus 1ex minus .2ex}{1.5ex plus .2ex}

% geometry
\usepackage{geometry}
\geometry{ headsep=20pt,
headheight=20pt,
left=21mm,
top=15mm,
right=21mm,
bottom=15mm,
footskip=20pt,
includeheadfoot}

% for using wrapfigures
\usepackage{wrapfig}

% header and footer
\usepackage{datetime}
\newdateformat{dmy}{%
\THEDAY.~\monthname[\THEMONTH] \THEYEAR}
\usepackage{fancyhdr}
\pagestyle{fancy}
\lhead{Noah Vogt \& Simon Hammer}
\chead{}
%\rhead{\dmy\today}
\lfoot{}
\cfoot{Gymnasium Kirschgarten}
\rfoot{Seite \thepage}
\renewcommand{\footrulewidth}{.4pt}

% fix figure positioning
\usepackage{float}

% larger inner table margin
\renewcommand{\arraystretch}{1.4}

% no paragraph indent
\setlength{\parindent}{0em}

% graphics package
\usepackage{graphicx}

\usepackage{multicol}

% use sans serif font
\usepackage{tgheros}
\usepackage{mathptmx}

% don't even ask what this is for, I have no idea (noah)
\usepackage{bm} %italic \bm{\mathit{•}}
\usepackage[hang]{footmisc}
\usepackage{siunitx}
\usepackage[font={small,it}]{caption}
\sisetup{locale = DE, per-mode = fraction, separate-uncertainty,   exponent-to-prefix, prefixes-as-symbols = false, scientific-notation=false
}
\newcommand{\ns}[4]{(\num[scientific-notation=false]{#1}\pm\num[scientific-notation=false]{#2})\cdot\num[]{e#3}\si{#4}}

% show isbn in bibliography
\usepackage{natbib}

\begin{document}

\begin{titlepage}

\vspace*{1cm}
	\centering
	
	{\huge\bfseries Eine Email-Client-App entwickeln \par}
	\vspace{0.5cm}
	{\Large Noah Vogt \& Simon Hammer\par}
	\vspace{17cm}

	{\large Geschrieben im Jahr 2021\par}
	
\end{titlepage}

\tableofcontents
\pagebreak

\section{Vorwort}

Ich (Simon) habe lange Zeit mich darüber aufgeregt, dass es keine Edubs-Mail-App gibt. Ich musste die Browserversion als Link auf meinem Smartphone Bildschirm 
einfügen und ich habe mich sehr darüber genervt, dass ich mich jedes mal beim öffnen der Browserversion neu anmelden musste. Als dann die Zeit kam mich für eine 
Maturarbeit zu entscheiden, wollte ich zuerst einen Edubs-Client erstellen, welcher für Schüler gut geeignet ist. 
Jedoch kannte ich bis dort nur Python und wollte mich mit meiner Idee mit Noah unterhalten, 
da ich wusste, dass er sicher auch etwas programmieren wird, weil er sich schon seit längerem mit Computern und der Programmiersprachen befasste. Er war der Meinung, ich könnte dies
doch gleich für jede Art von Email machen. Da ich mich noch nicht gut mit dem Programmieren auskannte, fragte ich ihn, ob er diese Maturarbeit denn mit mir machen würde.
Zu meiner Überraschung war er von der Idee überzeugt und wir einigten uns darauf, einen Email-Client zu erstellen, welcher zwar nicht für Schüler optimiert wurde, aber für 
den allgemeinen Gebrauch verbessert wurde. 

\section{Einleitung}
Das Kommunikationsmittel Email ist auch nach seinem fünfzigjährigen Jubiläum noch rege im Alltagsgebrauch vieler Leute in den meisten (industrialisierten) Ländern. Es könnte angenommen werden,
dass sich auch in den letzten zehn Jahren - nämlich seit dem Aufkommen der massentauglichen Smartphones - die Emailsoftware verbessert hat. Doch für das Smartphone war es für uns, die Ersteller dieser Maturarbeit, schwer einen guten Email Client zu finden. Diese Maturarbeit ist unser Teil, daran etwas zu ändern, wenigstens für uns.


\subsection{Ziel der Arbeit}

 was, wie und warum [machen wir das?]

\section{Konzept der Arbeit}

\subsection{Ziele der Funktionen}

Wir kennen alle die Funktionen eines Email-Clients und unsere App soll auch nicht viel mehr können als die meistens Apps. \\

Die App soll die Basisfunktionen eines klassischen Email-Clients erfüllen. Dazu gehören das lesen und schreiben von Emails,
öffnen und Anfügen von Anlagen, die Setzung einer Email-Signatur und das Erstellen und Speichern von Entwürfen. Ebenso soll es viele verschiedene Ordner geben, 
solche wie Spam/Junk oder ein Archiv Ordner. Dazu soll es möglich sein E-Mails verwalten zu können, zum beispiel indem die Ordner nach 
Datum des Empfangs oder dem Absender sortiert werden können. E-Mails könne markiert, gelöscht und weitergeleitet werden. Es soll eine Suchfunktion für jeden Ordner geben. \\

Ebenso soll es einen Account Manager geben, was voraus setzt, dass es möglich ist sich mit verschiedenen 
Accounts anzumelden und diese bei bedarf zu wechseln. Beim Anmelden einer E-Mail soll es dem Nutzer leicht gemacht werden mit eingebauten Konfigurationen für beliebte 
Emailprovider in der Schweiz. Darunter diese Anbieter: stud.edubs.ch, gmail.com, gmx.ch, outlook.com, yahoo.com, icloud.com, hotmail.com, web.de. \\

Ein Element, welches fast jede App Heutzutage hat, sind Pushnachrichten, welche auch eingebaut werden sollen. Dabei wird die Nachricht mit dem Absender und der Beginn der 
E-Mail angezeigt. 
Eine Funktion die nicht jeder Email-Client hat ist, dass die Gesprächsverläufe in der Struktur wie auf Reddit oder Thunderbird gezeigt werden. 
Dies wird ein weiteres Feature der App sein.
Wie es für leichte Email-Clients oft üblich ist, werden Bilder erst angezeigt, sobald es der Nutzer ausdrücklich möchte. Für dies 
wird eine Funktion in den Einstellungen  einfügen. Eine sehr praktische Funktion soll sein, dass E-Mail-Adressen gespeichert werden und 
beim schreiben einer neuen E-Mail direkt verfügbar sind und ausgewählt werden können. \\

Das letzte Feature ist, dass Links direkt in einem Browser geöffnet werden können. Die Einstellungen sollen das Farblayout der App ändern können, 
die Synchronisationsintervalle ändern können, Einstellungen an den Pushnachrichten ändern können, Kontaktlisten verwalten und Einstellungen zu Privatsphäre beinhalten. 

Ein Pluginmanager soll auch eingebaut werden, um weitere Funktionen, welche das Programm verlangsamen würden oder nicht für jedermann geeignet ist, hinzuzufügen können. 
Es gibt uns auch die Möglichkeit nach dieser Arbeit die App zu verbessern und auf uns an zu passen. \\ 

Wir haben uns viele Ziele gesetzt und dachten, dass wir auch mehr schaffen können. Weshalb dies nicht der Fall ist wird noch genauer betrachtet. 



\subsection{Vergleich mit Konkurrenz}
Disclamier/Note: Da bei dieser App einen Wert auf Sicherheit und Endnutzer-Freiheit gesetzt wird, wird sie dementsprechend nur mit Apps verglichen, welche auch Freie Software sind, nach Definition [der FSF??]. Somit fallen jegliche Proprietäre Produkte raus, da sie unseren Grundanforderungen eines Email Clients nicht entsprechen.\\


Wenn wir die meisten anderen quelloffenen, noch maintainten Open Source Email Clients anschauen fällt sofort auf, 
dass diese unglaublich überladen (bloated) sind. Selbst wenn im Internet nach einem möglichst simplen Email Client für Android gesucht wird, 
stösst man dabei meist auf Apps wie k-9 Mail, welche hunderttausende Zeilen Source Code besitzen.\\

Im Unterschied zur Konkurrenz soll diese App so programmiert werden, dass sie alle nötigen Grundfunktionen für einen Email Client auf dem Smartphone beinhaltet, 
aber schneller starten soll als die Apps der Konkurrenz, weniger Speicherplatz und Resourcen verbrauchen soll und nicht mit unnötigen Funktionen überladen sein.

\subsection{Quellcode Modell}
Um ein Programm zu programmieren, werden menschenlesbare Instruktionen in ein oder mehrere Textdateien geschrieben,
die dann in Sprache, welche die Computerhardware interpretieren und ausführen kann, übersetzt. 
Bei diesem Prozess geht die Lesbarkeit für den Menschen grösstenteils verloren. Um ein Programm ausführen können braucht der Nutzer also keinen Zugang zum Quellcode. Doch wenn er wissen will, was das Programm macht - es könnte ihn ausspionieren oder andere bösartige Sachen im Hintergrund machen - oder einfach das Programm verändern will, braucht er unbedingt Zugang zum Source Code.\\

Durch die zunehmende Kommerzialisierung und Massentauglichkeit der Computer haben sich aber monetäre und andere Anreize gebildet, den Source Code dem Nutzer nicht mehr zur Verfügung zu stellen. Das hat für den Nutzer verschiedene praktische Folgen. Aber vor allem ist er nicht mehr in Kontrolle seines Computers, Betriebssystems und Programms, sondern der Besitzer oder Copyrightholder des Programms. Das ein Programm einen Besitzer haben kann, ist ziemlich absurd und nur durch die Veränderungen der westlichen Copyright- Gesetze- und Kultur der letzten 150 Jahre [möglich?].\\

Doch auch bei Open Source Software gibt es verschiedene Lizenzmodelle welche gewählt werden können. Diese lassen sich gut ein zwei Kategorien unterteilen:\\

\paragraph{Permissive Licenses}
% ENG: These give you the right to run the program for ANY purpose, study the source code, change it and redistribute the changes. Their names come from the fact, that they are ``permissive'' when it comes to their few restrictions: They do not put of a lot of restrictions on the distribution of the source code, and often allow the software to be forked under ANY terms. This means it is possible to make your fork of a permissive-licensed program proprietary.\\
Mit diesen Lizenzen hat man das Recht das Programm für JEDEN Zweck zu benutzen, den Source Code zu lesen, ändern und seine Änderungen zu veröffentlichen. Ihr Name kommt von der Tatsache, dass sie ``permissiv'' sind, wenn es um ihre wenigen Einschränkungen geht: Sie schränken die Verbreitung des Quellcodes nicht sehr stark ein und erlauben es oft, die Software unter JEGLICHEN Bedingungen weiterzugeben. Das bedeutet es ist möglich, den Fork eines permissiv lizenzierten Programms proprietär zu machen.\\

% ENG: This means they do not protect their code from being taken and used in other projects, even proprietary, without any giving back, which is a definite drawback and the reason people insult these licenses as ``cuck licenses'' as you ``write proprietary code for free for big tech without increasing anyone's computer usage freedom''. A positive point of these licenses however is their simplicity: The MIT license for example only consists of 20 SLOC, while a Copyleft license as the GPL v3 uses more than 600 SLOC and its jurisdictional jargon is much harder to understand.\\
Das bedeutet, dass Sie den Code nicht davor schützen, in anderen Projekten, selbst wenn diese proprietär sind, verwendet zu werden, ohne etwas zurückzugeben zu müssen. Das ist ein eindeutiger Nachteil und der Grund, warum manche Leute diese Lizenzen als ``Cuck Licenses'' beschimpfen, da man ``When you release any code under a Cuck License, you are simply writing free commercial code for corporations that will inevitably use it against you. You might as well just actually get a job with them so you can get paid for what you do instead of just getting cucked.''. Ein positiver Punkt von diesen Lizenzen ist aber ihre Einfachheit: Die MIT-Lizenz zum Beispiel besteht aus nur 20 SLOC, während eine Copyleft Lizenz wie die GPL v3 aus mehr als 600 SLOC und gewissem Juristenjargon besteht und daher viel schwerer zu verstehen ist.\\

\paragraph{Copyleft Licenses}
% ENG: This is an obvious play on the word ``Copyright'', as it tries to invert its effects. As with permissive licenses you can run the program for ANY purpose, study, change the source code and redistribute the changes. But the trick comes with their restrictive redistribution terms: You can ONLY redistribute the program if your version provides the same user freedoms. This is usually done in only allowing redistributing using the same license.\\
Wie bei permissiven Lizenzen können Sie das Programm für JEDEN Zweck ausführen, den Quellcode lesen, ändern und die Änderungen zu veröffentlichen. Der Trick liegt jedoch in den restriktiven Redistributionsbedingungen: Sie dürfen das Programm NUR verbreiten, wenn Ihre Version die gleichen Freiheiten bietet. Dies geschieht in der Regel dadurch, dass die Weitergabe nur unter der gleichen Lizenz erlaubt ist.\\

Der Name ``copyleft'' ist eine offensichtliche Anspielung auf das Wort ``Copyright'', da es versucht, dessen Konzept genau umzukehren: Anstatt die nahezu weltweit vorhandenen Kopierrechtsgesetze dazu zu verwenden, Nutzerfreiheiten einzuschränken, werden Copyleft-Lizenzen dazu eingesetzt diese zu bewahren und erweitern.\\

% ENG: To provide such protection of the user freedoms in future forks, there needs to be a lot of jurisdictional jargon in the license which makes it significantly longer and harder to understand for the average reader. But most of the people in Free Software who care about user freedom say this is the premiere license for your free software projects, as it ensures that freedoms have to be granted forever and stopping proprietary code maker profiting from their written free code.\\
Wie bereits erwähnt, müssen solche Lizenzen eine Menge juristischen Jargon enthalten, um die vorhandene Gesetzeslage des Kopierrechts auszunutzen. Dies macht die Lizenz um einiges komplexer in Länge und an Verständlichkeit. Viele Unterstützer des Free Software Movements ziehen es aber trotzdem vor Copyleft Lizenzen zu verwenden, sofern möglich, da sie sicherstellen, dass die Freiheiten für immer gewährt werden müssen und die Hersteller von proprietärem Code daran hindert, von ihrem geschriebenen freien Code zu profitieren.\\


\subsection{Philosophie (suckless)}

``Many (open source) hackers are proud if they achieve large amounts of code, because they believe the more lines of code they've written, the more progress they have made. The more progress they have made, the more skilled they are. This is simply a delusion.''\\

% ENG: (Nowadays) a lot of Open Source Developers pride themselves in writing software with much features that cater to the non-technical enduser. This results in having a large codebase, which gets bigger and bigger with every release. This makes it harder to maintaining the ever growing codebase, more and more bugs occurs, security and (most importantly) performance struggles under these conditions. This degrades the quality of software technology as it is the mainstream narrative to ``save time and money''.\\

Eine grosse Anzahl Open-Source-Entwickler stolz darauf, Software mit scheinbar möglichst vielen Features zu schreiben. Dies führt zu einer großen Codebasis, die mit jedem Release größer und größer wird. Dadurch wird es schwieriger, die ständig wachsende Codebasis zu unterhalten, es treten immer mehr Fehler auf, die Sicherheit und - vor allem - die Leistung leidet unter diesen Bedingungen. Dies führt zu einer Verschlechterung der Code-Qualität der Software, da es die gängige Praxis ist, ``Zeit und Geld zu sparen''.\\


% ENG: This is where the \textit{suckless philosophy} comes in place: It aims at making software with simplicity in mind: Less source lines of code to not render the project unmaintainable in similar way as mentioned above. This way of programming is a lot more difficult, but the struggle is most of the time worth it. This coding philosophy also incentives (quality) code rewrites - which happens a lot less with bloated software counterparts - that gives the user more alternatives to choose from.

Hier kommt die Philosophie des \textit{Suckless Philosphy} ins Spiel: Sie zielt darauf ab, Software mit Hinblick auf Einfachheit zu entwickeln: Weniger Quellcodezeilen, um das Projekt nicht in ähnlicher Weise unmaintainable zu machen, wie oben erwähnt. Diese Art zu programmieren ist viel schwieriger, aber der Aufwand ist es meistens wert. Diese Programmierphilosophie bietet auch Anreize für das qualitativ hochwertige Neuschreiben von Programmen - was bei der sogenannter \textit{Bloated Software} viel seltener vorkommt - und gibt dem Benutzer mehr Alternativen zur Auswahl.



\subsubsection{Hintergründe, Technologisch, UNIX, KISS}
\subsection{Lizensierung}
% ENG: The differences of different source code models and their licenses have been already discussed in this paper (?). The reason the GNU General Public License Version 3 (short: GPL v3) was chosen because it is one of the most popular and strongest copyleft licenses that suits the application.

Die Unterschiede zwischen verschiedenen Source-Code Modellen und deren lizenzirrung wurden bereits besprochen. Für dieses Softwareprojekt wurde sich entschieden die \textit{GNU General Public License (GPL)} der Free Software Foundation zu verwenden, da diese eine der bekanntesten, strengsten und somit effektivsten Copyleft-Lizenzen ist.

\subsubsection{Hintergründe, philosophisch, technologisch}

\section{Hilfsmittel}

\subsection{Hardware}

\subsubsection{Smartphones}
% ENG: We used different Models of Smartphones running different variants of the android operating system to install and test our app. It was also used to compare it and benchmark it against our competition (the already existing mail clients on android).

Es wurden verschiedene Smartphone Modelle genutzt um die App auf verschiedenen Androidversionen testen zu können. Sie wurden auch genutzt um unsere App mit der Konkurrenz zu vergleichen und die 
Konkurrenz zu begutachten. 

\subsubsection{PC / Laptop}
% ENG: The usage of computers was to the actual work of creating the software, documentation and the text of this paper (?).

Der Gebrauch von Computern wurden gemacht um die Software, die Dokumentation und dieses Maturschreiben zu erstellen. 
\subsection{Software}

Aufgrund dessen, dass ein umfassendes Programm entstehen soll, wird auch Gebrauch von einigen anderen Programmen, Bibliotheken und sonstigen Tools gemacht. In den nachfolgenden Seiten wird beschrieben welche Programme genutzt werden, wieso diese ausgewählt wurden und wie der Umgang mit Ihnen war.  


\subsubsection{Programme}

\paragraph{Version Control System}

% ENG: For developing software above a certain level of complexity, it helps to use a so called VCS (Version Control System). This is a software/system that manages the versioning of software. This is useful so you don't have to for example create MANUALLY a new folder for every new version and copy the code with the new changes in there. This would be not very user friendly, wasteful of disk space and overall very inefficient. These systems also make it easy to synchronise projects to a remote server to allow for easy collaboration on multiple devices and for different developers.\\
Für die Entwicklung von Software ab einer gewissen Komplexität ist es hilfreich, ein so genanntes VCS (Version Control System) zu verwenden. Dabei handelt es sich um eine Software/System, das die Versionierung von Software verwaltet. Das ist nützlich, damit man nicht für jede neue Version manuell einen neuen Ordner anlegen und den Code mit den neuen Änderungen dorthin kopieren muss. Das wäre nicht sehr benutzerfreundlich, würde Speicherplatz verschwenden und wäre insgesamt sehr ineffizient. Diese Systeme erleichtern meist auch die Synchronisierung von Projekten mit einem Webserver, um eine einfache Zusammenarbeit auf mehreren Geräten und für verschiedene Entwickler zu ermöglichen.\\


\subparagraph{Git}
% ENG: It also allow for additional advanced features like branching a repository, which means developing multiple features separately from each other to then later merge the changes together, when each of the features are working as expected.
Git ist das für dieses Projekt verwendete VCS. Es ist das meistverbreitete Tool unter Softwareentwicklern und dabei ermöglicht auch zusätzliche fortgeschrittene Funktionen wie das \textit{Branching} eines Repositorys, was bedeutet, dass mehrere Funktionen getrennt voneinander entwickelt werden, um dann später die Änderungen zusammenzuführen, wenn jede der Funktionen wie erwartet funktioniert.\\

\subparagraph{Github}

% ENG: This is our remote git server. But is is more than just a git server, it is a popular platform with over 50 million (software) repositories that people search to find interesting software. So we can use this to make our software more searched and popular (but this is not considered of great importance).
Das ist der hier verwendete Git Server. Aber eigentlich es mehr als nur ein einfacher Git-Server, denn es ist auch eine populäre Plattform mit über 50 Millionen Software Repos welche gerne durchsucht werden, um interessante Software zu finden. Das kann also benutzt werden dieses Softwareprojekt berühmter zu machen, was aber nicht von hoher Wichtigkeit erachtet wird.


\paragraph{IDE}
 % eng IDE stands for ``Integrated Development Environment''. This means a program where the coder edits, debugs the code. It provides more features than just editing text, but also advanced autocompletion features and warnings/suggestions that the programmer should do something differently because of various reasons of bad programming practice. These programs have a lot of great features, but are often buggy, resource hungry and slow, like our choice of IDE: Android Studio.
IDE steht für ``Integrated Development Environment''. In diesem Programm editiert und debuggt der Programmiere seinen Code. Solche Software verfügt dabei um mehr Features als nur Texteditierung, sonden auch nützliche Autocompletion Features und gibt Warnungen oder Vorschläge, was der Programmier besser machen sollte. Solche Programme haben zwar viele nützliche Features, wovon wir nur einen Bruchteil kennen oder brauchen, sind aber oft buggy, resourcenhungrig und langsam. So auch die IDE Wahl dieses Projekts: Android Studio. Diese IDE wurde auch gewählt da sie stark optimiert und extra für die Androidentwicklung gemacht wurde.

% \paragraph{Texteditor}
% Text editors to have less ``advanced'' coding-related features, so they are not optimal for big projects with lots of files and graphical content.

\paragraph{Android Emulator(s)}
% ENG: An emulator is a program that simulates an device/operating system so that you can run inside another operation system. We use an Android Emulator inside of our GNU/Linux running on our computer, so that we do can test our app without always having to transfer it to an actual phone running android natively.

Ein Emulator ist ein Programm, welches ein Gerät oder Betriebssystem simuliert, sodass man es innerhalb eines anderen Betriebssystems laufen lassen kann. Für dieses Projekt wurde von Android Emulatoren Gebrauch gemacht, um die App gleich auf dem Computer testen zu können, wo man gerade am programmieren ist. So muss nicht ständig die App auf ein Smartphone transferiert werden, wo Android nativ läuft, nur um die App austesten zu können.

\paragraph{Compiler}
gradle, maven

\paragraph{Installer / Packaging}
% ENG: When we compiled our app using gradle, we get an .apk file as its output. APK stands for Android Package, which is the Package Format used for installing most of the software in Android. However, there are to sorts of apk that are compiled: a debug apk and a release apk. The debug version has extra features compiled in that help auditing bugs and new features not ready for using in the app available to users. The release Version has the current version number (?) added to it and is what you need to distribute of your working version to your users.
Nachdem die App mit \textit{Gradle} kompliert wurde, erhält man eine .apk-Datei als Output. APK steht für Android Package, was das Packaging Format ist um Software zu installieren auf Android. Dabei kann man die App aber auf zwei Arten komplilieren: eine \textit{debug APK} oder eine \textit{release APK}. Die Debug Version bekommt extra Features hineinkompliert welche dem Entwickler helfen Bugs zu finden und zu prüfen. Diese Version wird verwendet um neue Features auszutesten, welche noch nicht bereit sind für die Nutzer. Die Release Version enthält die derzeitige Versionsnummer und ist das was man braucht, um eine funktioniernde Version der App den Nutzern zur Verfügung zu stellen.

\subparagraph{signing keys}
% ENG: Nearly all Android Systems are configured to only allow application that are signed with a key from the developers. This is done for various reasons like security, compatibility checks and other (TODO: search more on this). So before we can install our app on 99\% of Smartphones, we first have to sign it with our cryptographic key we have created. Of course, this is only needed for actual releases and not for using debugging version e.g. in your android emulator.
Fast alle Android-Systeme sind so konfiguriert, dass sie nur Anwendungen zulassen, die mit einem Schlüssel des Entwicklers signiert sind. Dies geschieht aus verschiedenen Gründen wie Sicherheit, Kompatibilitätsprüfungen und anderen (TODO: mehr dazu suchen). Bevor wir also unsere App auf 99\% der Smartphones installieren können, müssen wir sie zunächst mit einem kryptografischen Schlüssel signieren, der von den Entwicklern dieser App erstellt wurde. Dies ist natürlich nur für die tatsächliche Veröffentlichung erforderlich und nicht für die Verwendung der Debugging-Version, z.B. in einem Android-Emulator.

\subparagraph{Android Debugging Bridge}
% ENG: This is a software tool to interact with android devices that are connected to a computer via usb. With it you can do useful thinks like screenshots, screen recordings, transferring files to and from your computer and installing / removing packages.

Dies ist ein Software-Tool für die Interaktion mit Android-Geräten, die per USB mit dem Computer verbunden sind. Damit kann man nützliche Sachen wie Screenshots, Bildschirmaufnahmen, das Übertragen von Dateien und die (De-) Installation von Apps, resp. Softwarepaketen.

\paragraph{Open Source Programme}

Beim Programmieren kann es sehr Hilfreich sein Programme zu haben welche, ähnliche Funktionen haben wie das Programm welches entstehen soll. 
Solche Vorlagen können beliebig getestet und verändert werden. Simon hatte zu beginn Schwierigkeiten Java zu nutzen, um Programme zu schreiben, da er noch nicht viel 
Erfahrung mit dem Programmieren hatte. Um sich mit der Art der Sprache und des Programmierens vertieft auseinander zu setzen, begann er Email-Apps, welche Open Source
waren, genauer zu Betrachten. Im folgenden Text werden wir diese Programme aufführen und zeigen für was wir sie gebraucht haben. 

\subsubsection{Librarys}
(bedeutung im glossar erklärt)

\paragraph{RecyclerViewer}

Der Recyclerviewer ist ein Container in welchen Daten gepackt werden. Er wird dem Layout hinzugefügt und hat eine grossen Vorteil gegenüber Listen. 
Eine Liste wird einmalig erstellt und komplett generiert. Das heisst es gibt Container, welche existieren, aber nicht auf dem Bildschirm angezeigt werden. 
Diese Container brauchen aber dennoch Speicherplatz, sind aber sinnlos. Hingegen der Recyclerviewer generiert nur so viele Container wie auf dem Bildschirm angezeigt werden können. 
Die Container, welche beim Scrollen am Bildschirm ende ankommen werden sogar wieder verwendet, mit neuen Information gespeist und am anderen Ende des Bildschirms angezeigt.

Bild von RecycleView

Ich (Simon) habe viel Zeit in einem template für den Recyclerviewer verbracht, da mir viele Internetseiten nicht helfen konnten. Es waren die ersten Schritte um 
richtig zu verstehen wie eine App funktioniert und wie sie Aufgebaut ist. Ich habe mich davor schon informiert jedoch wurde es mir dort richtig klar. Ich habe gelernt
wie Behälter über einen Key von Java Files aufgerufen werden und mit Daten gespeist werden. Ebenso habe ich verstanden, weshalb in Programmen, basierend auf Java, viele 
Klassen erstellt erstellt werden müssen, weil jeweils nur einmal die Variablen, Funktionen und Konstruktor einer andere Klasse implementiert werden können. Das heisst, 
wenn eine Klasse zwei Funktionen aus zwei unterschiedlichen Klassen verwendet werden soll, muss eine der beiden Klassen die andere Klasse implementieren. 

\subsection{Recherche Tools / Quellen}
\subsubsection{Internet}
% ENG: In the world wide web we use search engines like DDG (duckduckgo.com), the Arch Linux Wiki, Wikipedia and more programming-related websites like SOV (stackoverlow.com).
Um das Netz besser durchsuchen zu können, wurden Suchmaschinen wie DDG (duckduckgo.com), das Arch Linux Wiki, Wikipedia und programmierbezogene Internetseiten wie SOV (stackoverflow.com);
\subsubsection{Bücher, Artikel}
% ENG: We used different books and articles physically and digitally sometimes for the programming/coding process, but mostly for the text of this paper.
Es wurde auch Nutzen gemacht von Büchern und Artikeln, in digitaler und physischer Form, ein wenig für den Programmierprozess, aber hauptsächlich für den Maturarbeitstext.

\subsection{Dokumentation(-stools)}
\subsubsection{Latex}
% ENG: To write more formatting-demanding papers like in this paper it was made use of the typesetting engine LaTeX. This is a highly extensible and efficient way for getting your paper to look just like you want it. Also professionals use it to format a big majority of published books since \~ 20 years up to this day(TODO: look up on facts/statistics for this).
Um formatierungsintensivere Texte wie diese Arbeit zu schreiben, wurde das Satzprogramm LaTeX verwendet. Dies ist ein äußerst erweiterbarer und effizienter Weg, um eine Arbeit so aussehen zu lassen, wie man es exakt verlangt. Auch professionelle Fachleute verwenden es, um eine große Mehrheit der veröffentlichten Bücher seit etwa 20 Jahren bis zum heutigen Tag zu formatieren (TODO: Fakten/Statistiken dazu nachschlagen).
\subsubsection{Pandoc}
% ENG: This is program to convert from one document format to another. This is useful to write a document in format with very simple syntax (e.g. markdown) to convert it to a format that allows much better formatting (e.g. LaTeX) without having to use the more sophisticated syntax of the latter format. This is often used to write documentation or short texts with the minimal amount of time and effort to do so.
Dies ist ein Programm zur Konvertierung von einem Dokumentformat in ein anderes. Dies ist nützlich, um ein Dokument in einem Format mit sehr einfacher Syntax (z.B. Markdown) zu schreiben und es in ein Format zu konvertieren, welches eine viel bessere Formatierung ermöglicht (z.B. LaTeX), ohne die anspruchsvollere Syntax des komplexeren Formats verwenden zu müssen. Dies wird häufig verwendet, um Dokumentationen oder kurze Texte mit minimalem Zeit- und Arbeitsaufwand zu schreiben. So auch bei dieser Arbeit.

\section{Programmstruktur}

\subsection{Sicherheit / Security (Features)}
% ENG: Considering that on 99\% of consumer smartphones the users do not even have root access (but which the ``owner'' of the smartphone software do have), people that are concerned with privacy or security would not use their important mailboxes on their smartphone anyway, but rather on their better secured computer running a Free Software Operating System etc.\\
In Anbetracht der Tatsache, dass auf 99\% der Consumer-Smartphones die Benutzer nicht einmal Root-Zugriff haben (den aber der ``Besitzer'' der Smartphone-Software hat), würden Leute, die auf Privatsphäre oder Sicherheit bedacht sind, ihre wichtigen Mailboxen ohnehin nicht auf ihrem Smartphone benutzen, sondern eher auf ihrem besser gesicherten Computer, auf dem ein reines Freies-Software-Betriebssystem läuft usw.\\

% ENG: Also taking in mind that the email protocol was written in a time with a still very limited access to computers, privacy and security where not in mind of its creators at the time is quite understandable. Why it still lasted to this day in this state is compromised on one side by networking effects and on the other side that simply most people neither know to technical detail nor care.\\
wenn man bedenkt, dass das E-Mail-Protokoll in einer Zeit geschrieben wurde, in der der Zugang zu Computern noch sehr eingeschränkt war und Datenschutz und Sicherheit nicht im Sinne seiner Schöpfer waren, ist das durchaus verständlich. Warum es sich bis heute in diesem Zustand gehalten hat, liegt zum einen an den \textit{Networking Effects} und zum anderen daran, dass die meisten Menschen die technischen Details einfach nicht kennen und/oder sich nicht darum kümmern.\\


% ENG: Nowadays it is possible to encrypt to content (body) of an email, but not the metadata, here the specification of the email protocol is to blame. This and just the fact that is was not BUILT as a secure or private way of messaging, makes emails not useful for these kinds of conversation. If you want to send your friends new plans on overtaking the world and establishing a global catholic monarchy without wanting anyone to know, your are a fool if you use email for that(granted, this is an over exaggerated example, but I hope the idea is clear).\\
Heutzutage ist es möglich, den Inhalt (Text) einer E-Mail zu verschlüsseln, nicht aber die Metadaten; hier ist die Spezifikation des E-Mail-Protokolls schuld. Dies und die Tatsache, dass es nicht als sichere oder private Art der Nachrichtenübermittlung konzipiert wurde, macht E-Mails für diese Art von Konversation unbrauchbar. Wenn Sie Ihren Freunden neue Pläne für die Übernahme der Welt und die Errichtung einer globalen katholischen Monarchie schicken wollen, ohne dass jemand davon erfährt, sind Sie ein Narr, wenn Sie dafür E-Mails verwenden (zugegeben, das ist ein übertriebenes Beispiel, aber ich hoffe, der Gedanke ist klar).\\

% ENG: So the conclusion for our application project is to not bloat up our app with hard-to-use security functions that just bloat the app and codebase unnecessarily. However we use sane security-focused default now further explained.
Die Schlussfolgerung für unser Anwendungsprojekt ist also, unsere Anwendung nicht mit schwer zu verwendenden Sicherheitsfunktionen aufzublähen, die die Anwendung und die Codebasis nur unnötig aufblähen. Es sollten jedoch vernünftige, auf Sicherheit ausgerichtete Standardeinstellungen genutzt werden, die u.a. jetzt näher erläutert werden.

\subsubsection{PRIVATE MODE, Sandbox}
% ENG: When storing user settings (and data?), you can choose between different permission modes in android. We choose PRIVATE MODE, which means that apps with user permissions  can not see its content. But Google, root users, and apps with root access can easily bypass this restriction of the android permission system.\\
Wenn man Nutzereinstellungen- und Daten speichert in Android, kann der Entwickler zwischen verschieden Berechtigungsmodi wählen. In diesem Project wurde der sogenannte \textit{PRIVATE MODE} gewählt. Dieser bewirkt, dass andere Apps mit Nutzerberechtigungen (also a alle von einem Nutzer installierten Apps) den Inhalt dieser Daten nicht sehen können. Aber Google, Root-Nutzer und Apps mit Root-Zugriff können diese Einschränkung des \textit{Android Permission Systems} ohne weitere Probleme umgehen.\\

% ENG: To prevent this type of exploitation we would have to encrypt ALL sensitive user data with an encryption key. But when there is a keylogger installed, even this is not safe. So considering that there is not a single android phone free from proprietary operating system and firmware code(when connecting to the internet via WiFi), this feature idea has been rejected as out of place and to make our codebase simpler.
Um solch eine Art der Ausnutzung zu verhindern, müsste man ALLE sensiblen Benutzerdaten mit einem Verschlüsselungscode verschlüsseln. Aber wenn ein Keylogger installiert ist, ist selbst dies nicht sicher. In Anbetracht der Tatsache, dass es kein einziges Android-Telefon gibt, das frei von proprietärem Betriebssystem- und Firmware-Code ist (wenn es sich über WiFi mit dem Internet verbindet), wurde diese Funktionsidee als fehl am Platz und zur Vereinfachung unserer Codebasis verworfen.

% TODO: write about sandbox (and its 'lies') + maybe delete the last schizo paragraph

\subsection{Code Kompaktheit}
% ENG: This is something that we have absolutely achieved, we used so much less SLOC than any of the competing email clients on android.(TODO: compare use of libraries, and if we used more or less than the other apps) This is also very crucial to stand with our initial goal and the suckless philosophy. While we used \~ 4000 SLOC, other apps use a whopping 300'000 SLOC.
Das ist etwas, was im Laufe diese Projekts ziemlich gut eingehalten wurde: Es wurden viel weniger SLOC geschrieben als bei den konkurrierenden Email Clients auf Android. (TODO: compare use of libs, and if we used more or less than the other apps). Das ist ein sehr wichtiger Massstab um das initiale Ziel der \textit{Suckless Philosophy} so gut wie möglich zu erreichen. Während dieses Projekt nur \~ 4000 SLOC Zeilen Code aufzuweisen hat, sind es bei anderen Apps gar über 300'000 SLOC.
\subsubsection{Maintaining}
% ENG: Our codebase should be very easy to maintain: The main things you would have to do is check if the dependencies are up to date and if they can be considered deprecated.\\
Die Codebasis dieses Projektes sollte sehr einfach zu maintainen sein: Die wichtigsten Punkte dabei sind, dass der Maintainer schaut, dass die Dependencies auf der neusten Version sind und ob sie als obsolet oder veraltet angesehen werden müssten.\\

% ENG: Even if this project gets abandoned by their original developers - or as some people call that: when it get orphaned - and if someone finds it and likes it, he can easily read through it and understand the code without too much effort needed of understanding the codebase. Would the same be the case for bloated programs like thunderbird? I don't think so either.
Selbst wenn das Projekt aufgegeben werden sollte bei seinen ursprünglichen Entwicklern - oder wie manche sagen: ``the project/package gets orphaned'' (TODO: source pls) - und wenn jemand das Projekt findet und es ihm gefällt kann er dann recht einfach und schnell die Codebase durchlesen und sie ohne allzu grossen Aufwand verstehen. Wäre das auch der Fall bei bloated Software wie thunderbird? Das glaube ich auch nicht.

\subsubsection{less bug/error-prone}
% ENG: This is thanks to the suckless nature of our coding philosophy, but also our execution. It could be the case however that bad practices may have been used as the initial project owner did not have a lot of experience of java or android coding.
Es ist grösstenteils der suckless-basierten Codingphilosophie zu verdanken, aber auch der Umsetzung. Es könnte jedoch der Fall sein, dass schlechte \textit{Coding Practices} verwendet wurden, da die beiden initialen Entwickler dieses Projekts nur eine beschränkte Erfahrung mit der Androidentwicklung mit der Java-Programmiersprache hatten.

\section{Umsetzung}

\subsection{Informationsbeschaffung}

Zu beginn der Arbeit konnten Noah und Simon noch kein Java. Noah kannte sich gut aus mit Programmiersprachen und konnte es deshalb mit Java ruhig angehen.
Simon hingegen wusste, dass er noch viel zu lernen hatte. Er erhoffte sich die Grundlagen im Informatik Unterricht zu lernen. 
Simon lernte auch einige dinge, jedoch ging ihm das zu langsam und er schaute sich nach anderen Lernmöglichkeiten um. 
Zu diesem Zeitpunkt konnten einige Schüler an einem Programm mitmachen, bei welchem sie in die Uni gingen und gewisse 
Vorlesungen besuchen konnten. Ein Klassenkamerad von Simon konnte von diesem Programm profitieren und besuchte Informatik. 
Dort lernte er in einem unglaublichen Tempo Java, weshalb Simon seinen Klassenkameraden fragte ob er seine Unterlagen ausleihen konnte. 
Sein Klassenkamerad war so freundlich und übermittelte ihm die Unterlagen. So konnte Simon auf effektive weise Java lernen. \\

\textbf{ Noah wie hesch du Java glernt?}
Persönlich hatte Noah schon Erfahrung mit verschieden Programmiersprachn. Am ähnlichsten zu Java waren dabei meiner Einschätzung nach C und C++, welche eine teilweise ähnliche Sytax gegenüber Java aufweisen. Wirklich etwas mit Java gemacht hat er bei der Pluginprogrammierung für Minecraft und im EF (Ergänzungsfach Informatik) am GKG (Gymnasium Kirschgarten).\\

Als nächstes ging es darum wie eine App aufgebaut ist und wie zwei Personen Zweitgleich an einem 
Programm arbeiten konnten. Dafür musste sich Simon über Git/Github informieren. Noah hatte ihm ein Video empfohlen, welches 
Simon studieren soll und er soll kleine Übungen mit Git ausprobieren um ein Gefühl dafür zu bekommen. Die beiden Informierten 
sich über verschiedene Plattformen wie eine App funktioniert und wie sie aufgebaut ist. \\ 

Es ging langsam voran und die beiden begannen mit technischen Einstellungen. Als erstes 
wurde Android Studio  und eine ADB installiert. Noah hat sogar eine Anleitung zur Installation von ADB geschrieben. 
Um diese auszuprobieren und die Grundlage zu schaffen auf welcher Simon und Noah die nächsten paar Monaten arbeiten werden, haben sie ein
Testprogramm erstellt namens "Hello World". Es war ein ganz simples Programm das nur zum Testen der Infrastruktur diente. 
Sie haben ein Gradle.build file erstellt, dies dauerte erstaunlich lange. Später erkannten sie das diese Verhalten sehr üblich für 
Gradle ist. Nach einiger Zeit haben es beide geschafft ein apk file zu besitzen, welches erfolgreich die "Hello World" app beinhaltete und ausführt. 
Anschliessend erstellte Noah ein GitHub repository in welchem die beiden das Programm teilen werden. \\

\textbf{ https://www.youtube.com/watch?v=2sjqTHE0zok} nur um später zu makieren.

\subsection{Interface}

Noah begann dann dieses Programm leicht auszubauen und fügte einige Interface Inhalte hinzu. 

\begingroup
\setlength{\intextsep}{10pt}
\setlength{\columnsep}{15pt}

\begin{wrapfigure}{r}{8cm}
\centering
\includegraphics[scale=.15]{media/drawer.png}
\caption{Drawer Navigation Menu}
\end{wrapfigure}
%I then made the app more functional, so that you have a base GUI with a drawer, a menu in the bottom and in the drawer navigation menu you can tap on the «Add Email» Button and a popup window will come up asking you for name, email and password. Even the save and cancel button work. Now we only need a functionality to save this information to a string somewhere in the main activity.\\

Die App beinhaltete ein Burgermenü mit drei Ordnern und einem Knopf, welcher den Nutzer zu einem Popupwindow führte, in welchem es so aussah als würde es möglich sein sich anzumelden.  
Die Basis für die App wurde gelegt und die beiden entschieden sich mit dem Interface anzufangen. Für Simon ein guter Anfang sich mit dem 
richtigen Programmieren zu befassen und sich einleben zu können. 
Simon begann mit dem RecyclerViewer

\subsection{Recyclerviewer}

Simon begann mit der Recherche zum Recyclerviewer. Er war viel im Internet und fand schnell
einige Beispiele, welche er ausprobierte und für seinen Nutzen gestaltete. Ein Recyclerviewer ist nicht sehr kompliziert
hat aber gewisse kniffe auf die Simon traf.

Ein Recyclerviewer ist in fünf grundlegende Teile aufgeteilt. 
1. Das recyclerview Objekt, welches ein Container ist und in das User Interface eingebaut wird. 
Es beinhaltet verschiedene Views, welche nochmals unterteilt werden können. In einer View wird in unserem 
Fall eine Nachricht eingebaut mir dem Absender, der Absende Zeit und einem Betreff. 

2. Der Layout manager. Er ist für die form einer einzelnen View verantwortlich. 
Der Layout manager kann auch wieder in drei Arten unterteilt werden. Der Linearlayout Manager sorgt für eine 
Horizontale oder vertikale unterteilung einer View. Hingegen führt der Gridlayout Manger zu einer horizontalen 
und vertikalen Unterteilung des View. Für den Email-Client bietet er die  
beste oberfläche um Nachrichten darzustellen. Es gibt nämlich noch den den Staggeredgridlayout Manager, welcher 
für eine versetzen unterteilung der View kann sorgen. Dies ist aber für eine Email-Client unbrauchbar. 

\textbf{ Drei bilder der Manager mit der View eifügen}

3. Der View Holder bietet die Möglichkeit jedes einzelne Raster der View mit einem Item auszustatten mit wiederum einer View

4. Der Adaper ist wohl eines der wichtigsten Teile des Recyclerviewer. Er sorgt für das erstellen des View Holder Objekt und
bindet auch die Daten aus der Database an den View Holder

5. Die Database ist das Herzstück dieser App und wir auch für den Recyclerviewer verwendet um die 
Views mit den richtig Informationen zu füllen. 

\subsection{Beispiele aus der Umsetzung}
\subsubsection{Bugs}
Erklärung woher, warum falsch, wie gelöst
GUI stuff -> onliner commit 

\subsubsection{Probleme / Hiccups}
RecyclerViewer, shit java IMAP libraries

\subsubsection{Kommunikation}
simon fragen, war Kommunikation zwischen uns gut?

\section{Einschätzung / Schlussfolgerung}

\subsection{Fremdeinschätzung}
\subsection{Selbsteinschätzung}
\subsubsection{was lief gut}
komm., vcs, java syntax (allg.)
\subsubsection{was lief schlecht}
******* java shit tier libraries
\subsubsection{was würden wir gleich machen}
As long as java is not deprecated in the future for android programming, we would still use it as it is the most ``native'' way of programming. The new java clone/fork of google, Kotlin, would be a worse choice in our eyes, as your are even more dependent on google code, which you are already by using android.
\subsubsection{was würden wir anders machen}
As java is much slower than language that can be compiled into native binaries we would try to use more C or C++ libraries to improve speed and portability. 


use more c libs for speed and less java (that's always a good thing)
\subsubsection{abschliessende persönliche Schlussfolgerung}

\section{Danksagung}


\end{document}
